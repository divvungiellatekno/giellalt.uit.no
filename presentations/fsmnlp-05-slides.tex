\documentclass[landscape,english,11pt]{seminar} 

\def\everyslide{\sf}
\usepackage{babel}
\usepackage{ucs}
\usepackage[utf8]{inputenc}

\usepackage[T1]{fontenc}

\usepackage{hyperref}
\usepackage{graphics}
%\slidesmag{5}
\slideframe{none}

%\usepackage{pp4slide}
%\usepackage{pause}


\title{From Xerox to Aspell: A first prototype of a North Saami speller
 based on TWOL technology}

\author{Sjur Moshagen
, Børre Gaup
, \\Thomas Omma  
, Maaren Palismaa 
and Tomi Pieski\\ 
 The Saami Parliament, Norway
\and Trond Trosterud\\ University of Tromsø
\and \\
\and \scalebox{0.20}[0.20]{\includegraphics{samediggi.jpg}}
\and \scalebox{0.20}[0.20]{\includegraphics{logoWeb070.jpg}}
}

%\author{Børre Gaup\\ The Saami Parliament\\ Norway
%\and Sjur Moshagen\\ The Saami Parliament\\ Norway
%\and Thomas Omma\\ The Saami Parliament\\ Norway
%\and Maaren Palismaa\\ The Saami Parliament\\ Norway
%\and Tomi Pieski\\ The Saami Parliament\\ Norway
%\and Trond Trosterud\\ Faculty of the humanities\\ University of Tromsø
%\and \\
%\and \scalebox{0.20}[0.20]{\includegraphics{logoWeb070.jpg}}
%\and \scalebox{0.20}[0.20]{\includegraphics{samediggi.jpg}}}

%\textit{http://giellatekno.uit.no} }\\ 
\begin{document}
\begin{slide}

\maketitle


\newslide

\textbf{Aspell}

\begin{itemize}
\item To build a parser and disambiguator for North and Lule Sámi, and to make a practical spellchecker for the same languages.
\item ...
\end{itemize}


  The core analyser is written with the Xerox tools \texttt{twolc}, \texttt{lexc} and \texttt{fst}, and the disambiguator uses costraint grammar (\texttt{vislcg}).

\newslide

Main points:
- Source and tools: Xerox tools (fst)
- Target - a working speller: Aspell (one-level "automaton")
- limitations:

\end{slide}
\end{document}